	\begin{tabularx}{\textwidth}{|l|X|}
		\hline
		\textbf{Begriff} & \textbf{Erklärung}\\ \hline
		Robot & Roboter, der sich im zweidimensionalen Raum bewegen kann. Er wird als Fahrzeug verwendet und kann Personen transportieren.\\ \hline
		Position & Eine Position ist eine zweidimensionaler Punkt im Raum, zu dem der Roboter fahren kann.\\ \hline
		Destination & Eine Destination beinhaltet zusätzlich zu einer Position auch eine Geschwindigkeit. Dadurch kann der Server dem Roboter mitteilen, ob dieser vorsichtig oder schnell zu einer Position fahren soll. \\ \hline
		Task & Ein Task ist eine Kette von Destinations. Der Roboter führt stets einen Schritt (Destination) aus und wartet auf ein Kommando, um weiter zu machen. Ein typischer Task ist z.B. fahre schnell zu einer gegebenen Adresse und vorsichtig zum Krankenhaus.\\ \hline
		Server & Der Server ist eine einzelne, zentrale Instanz, die Tasks an die Roboter verteilt.\\ \hline
		Charger & Ladestation für den Roboter\\ \hline
		Destination & Ziele die der Roboter ansteuern kann.\\ \hline
		Obstacle & Hindernisse, die der Roboter erkennt und umfährt. Dies können
		entweder andere Roboter sein, mit denen der Roboter kommunizieren kann
		um eine Priorität festzulegen oder statische Hindernisse der
		Umgebung.\\ \hline
		Assistance & Helfer, die die Person in den Roboter laden.\\ \hline
		Patient & Wird von den Robotern abgeholt und in das Krankenhaus gebracht.\\ \hline
		Hospital & Das Krankenhaus, das mit dem System kommuniziert, um die Robots für Krankentransporte einzusetzen. \\ \hline
		Taxi-Customer & Kunde, der über eine App ein Taxi beim Server bestellt, abgeholt wird und zum Ziel transportiert wird.\\ \hline
		ServerCore & Zentrale Recheneinheit des Servers, auf der die Serversoftware und das Betriebssystem ausgeführt werden.\\ \hline
		RobotCore & Zentrale Recheneinheit des Roboters, auf der die Robotsoftware und das Betriebssystem ausgeführt werden.\\ \hline
	\end{tabularx}
