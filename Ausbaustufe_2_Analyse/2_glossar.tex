	\begin{tabularx}{\textwidth}{|l|X|}
		\hline
		\textbf{Begriff} & \textbf{Erklärung}\\ \hline
		Robot & Roboter, der sich im zweidimensionalen Raum bewegen kann. Er wird als Fahrzeug verwendet und kann Personen transportieren.\\ \hline
		Position & Eine Position ist ein zweidimensionaler Punkt im Raum, zu dem der Roboter fahren kann.\\ \hline
		Destination & Eine Destination beinhaltet zusätzlich zu einer Position auch zusätzliche Daten, die es ermöglichen weitere Parameter festzulegen. 
		Dadurch kann z.B. unterschieden werden, ob ein Roboter vorsichtig oder schnell zu einer Position fahren soll. \\ \hline
		Task & Ein Task ist eine Folge von Destinations, die ein Roboter nacheinander abarbeiten kann. Ein Beispiel für einen Task ist, dass der Roboter schnell zu einer gegebenen Adresse und vorsichtig zum Krankenhaus zurück fährt.\\ \hline
		Order & @fluepke\\ \hline
		Server & Der Server ist eine einzelne, zentrale Instanz, die Tasks an die Roboter verteilt und als Kommunikationsschnittstelle dient.\\ \hline
		Charger & Eine Position an der der Roboter seinen Akku aufladen kann.\\ \hline
		Obstacle & Hindernisse, die der Roboter erkennt und umfährt. Diese können
		entweder andere Roboter sein, mit denen der Roboter kommunizieren kann
		um eine Priorität festzulegen, oder statische Hindernisse der
		Umgebung.\\ \hline
		Assistant & Personen, die sich bei dem Patienten befinden und ihn auf den Roboter setzen oder legen.
		 Dazu gehören ebenfalls Mitarbeiter des Krankenhauses die den Patienten zur weiteren Behandlung abholen.\\ \hline
		Patient & Person, die in das Krankenhaus gebracht werden muss und sich an einer bestimmten Position befindet. 
		Sie wird von den Robotern abgeholt und in das Krankenhaus gebracht.\\ \hline
		Hospital & Das Krankenhaus, das mit dem System kommuniziert, um die Robots für Krankentransporte einzusetzen. \\ \hline
		Taxi-Customer & Kunde, der über eine App ein Taxi beim Server bestellt, abgeholt und zum Ziel transportiert wird.\\ \hline
		Customer & Personalisierte Generalisierung von Patient und Taxi-Customer, also Personen die von den Robotern transportiert werden.\\ \hline
		User & Generalisierung von Hospital und Taxi-Customer, insbesondere Akteure die Orders an den Server übermitteln können.\\ \hline
		ServerCore & Zentrale Recheneinheit des Servers, auf der die Serversoftware und das Betriebssystem ausgeführt werden.\\ \hline
		RobotCore & Zentrale Recheneinheit des Roboters, auf der die Robotsoftware und das Betriebssystem ausgeführt werden.\\ \hline
	\end{tabularx}
