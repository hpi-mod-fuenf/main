
\section{Zielbestimmung}
Die Aufgabe des Projektteams ist es, ein System zu entwickeln, das selbstfahrende Roboter in einer Stadt der Zukunft koordiniert. Sie sollen sowohl für einen Krankentransport- als auch einen Taxiservicedienst eingesetzt werden. 
In dieser Stadt existieren keine Straßen mehr. Damit verlangt es eine gewisse Komplexität zu erfassen und in die richtigen Bahnen zu lenken, um die Roboterdienste unfallfrei über den Raum der Stadtkarte zu koordinieren. Die größte Schwierigkeit stellt hierbei der Prozess dar das Navigationsverhalten der Roboter mit den richtigen Parametern einzustellen und zu modellieren. Weitere Probleme ergeben sich daraus, die verschiedenen Krankentransporter in einer Weise zu synchronisieren, dass keine gegenseitigen Behinderungen auftreten. 
Die Voraussetzungen und Spezifikation dafür zu erfassen, ist Ziel dieser Analyse – die sich elementar mit dem Fahr- und dem Aufladevorgang der Roboter und ihren Manövrierfähigkeiten gegenüber Hindernissen beschäftigt; im speziellen Fall des Krankentransporters zudem mit dem Entsenden zum Einsatzstandort, der Aufnahme von Patienten und der Kommunikation mit dem Hospital; und im Nutzungsgebiet des Taxiservice dem Grundbau eines Appdienstes, der eine Servicefläche bietet, um verfügbare Taxis anzuzeigen und an Kunden zu senden. \\

Als Ausgangspunkt stellt das Unternehmen eine unbegrenzte Anzahl von Transportrobotern zur Verfügung, die mit Aktoren, Sensoren und einem Ortungsgerät ausgestattet sind, und im Wechselspiel mit einem Server agieren. \\

Das Primärziel dieser Analyse ist es, dabei einen möglichst reibungslosen Ablauf der Kranken und Taxitransporte für menschliche Individuen zu ermöglichen. Davon würde die Zielgruppe, spezifisch die Bewohner der Stadt, nachhaltig profitieren, indem die Zahl der Verkehrsunfälle und Staus sowie die Dauer des Fahrens auf ein Minimum reduziert werden – somit Verletze stets rechtzeitig zur Behandlung im Krankenhaus eintreffen und Privatpersonen schnellstmöglich und flexibel jeden Ort der Stadt erreichen. Die Umsetzung des Systems birgt jedoch zahlreiche Herausforderungen, die im Verlauf dieser Analyse betrachtet werden. Falls die verschiedenen Hürden gemeistert werden, könnte langfristig ein einzelnes Unternehmen die gesamte Versorgungstransportinfrastruktur einer Großstadt übernehmen. Das System wird von sich aus selbstständig und vollständig autonom agieren. Unter Beachtung von Sicherheitsregeln, sollten damit keine weiteren Vorkenntnisse notwendig sein, um die Servicedienste zu nutzen. Gerade für Patienten sollten die Roboter den Fahrvorgang dabei mit kleinstmöglichem Aufwand und Belastung gestalten und für Taxiservicegäste den schnellstmöglichen und sichersten Weg anbieten. \\

\pagebreak
