
\section{Zielbestimmung}
Die Aufgabe des Projektteams ist es, ein System zu entwickeln, das selbstfahrende Roboter in einer Stadt der Zukunft koordiniert. 
Diese Roboter sollen für einen Krankentransportdienst eingesetzt werden. 
In dieser Stadt existieren keine Straßen mehr – umso komplexer wird es die Steuerung der Roboter über den Raum dieser Stadt zu koordinieren. 
Die größte Schwierigkeit stellt hierbei der Prozess dar, das Navigationsverhalten der Roboter mit den richtigen Parametern einzustellen und zu modellieren. 
Ein weiterer Teil der Aufgabe besteht darin, die verschiedenen Krankentransporter in einer Weise zu synchronisieren, dass keine gegenseitigen Behinderungen auftreten. 
Die Voraussetzungen und Spezifikation dafür zu erfassen, ist Ziel dieser Analyse – die sich mit dem Fahr- und dem Aufladevorgang der Roboter, im speziellen den Fahrvorgang mit Patienten, ihren Manövrierfähigkeiten gegenüber Hindernissen, dem Aufnahmevorgang und die Entsendung zum Standort von Patienten beschäftigt.\\

Im weiteren Verlauf des Projektes könnten weitere Gegebenheiten auftreten, an die man das System anpassen müsste. 
Alle hier getroffenen Annahmen, Modellierungen und Implementierungen könnten damit im Rahmen der späteren Anwendungen revidiert werden.\\

Als Ausgangspunkt stellt das Unternehmen eine unbegrenzte Anzahl von Krankentransportrobotern zur Verfügung, die mit Aktoren, Sensoren und einem Ortungsgerät ausgestattet sind, und im Wechselspiel mit einem Server agieren.\\

Das Primärziel dieser Analyse ist es, dabei einen möglichst reibungslosen Ablauf der Krankentransporte für menschliche Individuen zu ermöglichen. 
Davon würde die Zielgruppe, spezifisch die Bewohner der Stadt, nachhaltig profitieren, indem die Zahl der Verkehrsunfälle und Staus sowie die Dauer des Fahrens auf ein Minimum reduziert werden – und somit Verletze stets rechtzeitig zur Behandlung im Krankenhaus eintreffen würden. 
Die Umsetzung des Systems birgt jedoch zahlreiche Herausforderungen, die im Verlauf dieser Analyse betrachtet werden. 
Falls die verschiedenen Hürden gemeistert werden, könnte langfristig ein einzelnes Unternehmen die gesamte Krankentransportinfrastruktur einer Großstadt übernehmen. 
Das System wird von sich aus selbstständig und vollständig autonom agieren. Unter Beachtung von Sicherheitsregeln, sollten damit keine weiteren Vorkenntnisse für Patienten und ihre Begleiter notwendig sein, um den Krankentransportdienst zu nutzen. 
Gerade für Patienten sollten die Roboter den Fahrvorgang mit kleinstmöglichem Aufwand und Belastung gestalten.\\

\pagebreak
