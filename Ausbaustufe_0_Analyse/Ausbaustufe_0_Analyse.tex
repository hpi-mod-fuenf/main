\documentclass[11pt, a4paper]{article}
\usepackage[utf8]{inputenc}
\usepackage[ngermanb]{babel}
\usepackage[dvipsnames]{xcolor}
\usepackage{amsmath,amscd,amssymb}
\usepackage{enumerate}
\usepackage{tabularx,ragged2e,booktabs,caption}

\begin{document}
	\section{Zielbestimmung}
	\input{1-Zielbestimmung}
	
	\section{Produkteinsatz}
	\input{2-Produkteinsatz}
	
		\subsection{Beschreibung des Problembereichs}
		\input{2-1-Beschreibung_des_Problembereichs}
		
		\subsection{Glossar}
		\input{2-2-Glossar}
		
		\subsection{Modell des Problembereichs}
		\input{2-3-Modell_des_Problembereichs}
		
		\subsection{Beschreibung der Geschäftsprozesse}
		\input{2-4-Beschreibung_der_Geschaeftsprozesse}
		
			%\subsubsection*{Beschreibung zu Prozess-ID: Name des Geschäftsprozesses}
			
	\section{Produktfunktionen}
	\input{3-Produktfunktionen}
	
		\subsection{Use Cases}
		\input{3-1-Use_Cases}
		
		%\subsection{Beschreibung zu Use Case-ID : Use Case-Name}
		
			%\subsubsection*{Charakterisierende Informationen}
			
			%\subsubsection*{Szenario für den Standardablauf (Erfolg)}
			
			%\subsubsection*{Szenarien für alternative Abläufe\\ (Misserfolg oder Umwege zum Erfolg)}

			%\subsubsection*{Beschreibung des allgemeinen Ablaufes}
			
	\section{Produktumgebung}
	\input{4-Produktumgebung}
	
		\subsection{Systemumgebung}
		\input{4-1-Systemumgebung}
		
			\subsubsection{Hardwareumgebung}
			\input{4-1-1-Hardwareumgebung}
			
			\subsubsection{Softwareumgebung}
			\input{4-1-2-Softwareumgebung}

			\subsubsection{Ressourcenübersicht}
			\input{4-1-3-Ressourcenuebersicht}
					
\end{document}