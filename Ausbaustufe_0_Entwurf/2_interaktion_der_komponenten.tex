\section{Interaktion der Komponenten}
Auf Basis der Use Cases aus dem Analysedokument wird in diesem Kapitel die Interaktion der einzelnen Komponenten aus Kapitel 1 betrachtet. Dabei liegt der Fokus vor allem auf der Interaktion zwischen dem \emph{Server} und der \emph{RobotUnit}. Die Use Cases innerhalb der Komponenten werden in Kapitel 8 näher ausgeführt.\\


\subsection*{Interaktion bei Ausführung von \emph{4: Choose Robot}}

Die Auswahl eines \emph{Robots} läuft wie in Abbildung \ref{SequenzDiagrammInteraktion} beschrieben folgendermaßen ab: Bei einer neu eingehenden \emph{Destination} sendet der \emph{Server} mit \texttt{getTaskRating(task)} Anfragen an alle zur Verfügung stehenden \emph{RobotUnits} (im Diagramm sind beispielhaft zwei angeführt, der Aufruf findet asynchron statt). Die \emph{RobotUnit} führt dann \emph{Read Sensor} (Use Case 2) aus. Dabei sammelt sie alle notwendigen Informationen von ihren Hardwareschnittstellen, wie Ladestand und Nähe zum Ziel, die der \emph{Server} benötigt, um den bestmöglichen \emph{Robot} für das Ziel auszuwählen. Der \emph{Robot} wartet auf das Zusammentragen der Daten, also das Abschließen des internen Loop-Prozesses, bis er eine Nachricht mit den Informationen an den \emph{Server} zurücksendet; erst dann kann der \emph{Server} auf Grund der übermittelten Daten beurteilen, welcher \emph{Robot} am besten dazu geeignet ist, die \emph{Destination} zu erreichen und entsendet ihn zu den Zielkoordinaten (\emph{assign Task} – Use Case 5). Bei dem darauffolgenden Use Cases \emph{DriveToDestination} ist der \emph{Server} hingegen nicht beteiligt. Bei dem Use Case \emph{Charging} kommt es ausschließlich zur Wechselwirkung mit der \emph{ChargingStation}-Komponente.

\begin{figure}[H]
	\centering
	\includegraphics[width=0.9\textwidth]{img/2-chooseRobot}
	\caption{\emph{ChooseRobot}-Sequenzdiagramm}
	\label{SequenzDiagrammInteraktion}
\end{figure}
