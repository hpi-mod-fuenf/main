\section{Produkteinsatz}
\textcolor{blue}{\textit{In diesem Abschnitt wird der geplante Einsatz des zu entwickelnden Produktes beschrieben. Dies umfasst insbesondere die Systemumgebung in der das Produkt eingesetzt werden soll und die Zuordnung der Software zu dieser.
}}

\begin{figure}[H]
\centering
\includegraphics[width=0.75\textwidth]{../images/9_produkteinsatz.svg}
\label{Produkteinsatz}
\end{figure}

In diesem Abschnitt wird der geplante Einsatz des Systems beschrieben, wobei insbesondere auf die Systemumgebung, in der das Produkt eingesetzt werden soll, und die Zuordnung der Software zu dieser eingegangen wird.

Das Gesamtsystem besteht u.a. aus einem \emph{Server}, der mit mindestens einem \emph{Robot} verbunden ist, wobei zwischen \emph{Server} und \emph{Robot} über einen nicht näher spezifizierten IMessageHandler kommuniziert wird. \emph{Robots} können nicht mit anderen \emph{Robots} kommunizieren.

Auf dem \emph{Server} und den \emph{Robots} läuft ein \emph{Java Runtime Environement}, das dem Ausführen der entsprechenden Software dient.

\emph{Robot.jar} dient der Kapselung der verfügbaren Funktionen. Es werden auch die Interfaces zusammengefasst, die dem Ansprechen von \emph{IDrive}, \emph{INorthStar}, \emph{IDistanceSensor}, \emph{IBumper}, \emph{IBumperHandler} und \emph{IBattery} dienen.

\emph{Server.jar} funktioniert analog zu \emph{Robot.jar} und stellt alle grundlegenden Funktionen zur Verfügung.

Zusätzlich benötigte Funktionalitäten werden aus dem Common-Package importiert und sind sowohl bei \emph{Robots} als auch dem \emph{Server} verfügbar.
