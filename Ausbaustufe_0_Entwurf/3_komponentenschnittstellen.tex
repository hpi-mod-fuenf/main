\section{Komponentenschnittstellen}
Abbildung \ref{KomponentenschnittstellenDiagramm} zeigt die Schnittstellen der eingeführten Komponenten. 

Die Aufgabe des Datentyps \emph{SensorData} ist primär, die Koordination mit dem Server zu unterstützen, um festzustellen, inwieweit eine Zielposition gut erreicht werden kann. Dazu besitzt er als Attribute die Orientierungsrichtung im Koordinatensystem, den Batteriestatus und zuletzt Attribute der Datentypen \emph{Position} und \emph{Destination}. \emph{Position} besitzt wiederum die Koordinaten x und y, die einen beliebigen Punkt im Bereich des Einsatzgebietes des Roboters darstellen. Um Redundanz zu vermeiden sind \emph{Destination} und \emph{Position} über keine zusätzliche Klassendiagrammbeziehung verknüpft. Zielpunkte erben von \emph{Position} und existieren zur Spezifikation, um welche Art Zielpunkt es sich handelt: Also zum Beispiel eine allgemeine \emph{Destination} oder den \emph{Charger}.
	
	\begin{figure}[H]
		\centering
		\includegraphics[width=0.75\textwidth]{img/3-Zusatzklassen}
		\caption{Komponentenschnittstellen}
		\label{KomponentenschnittstellenDiagramm}
	\end{figure}
