
\section{Zielbestimmung}
Die Aufgabe des Projektteams ist es ein System zu entwickeln, das selbstfahrender Roboter in einer Stadt der Zukunft koordiniert, die für einen Krankenhaustransportdienst eingesetzt werden. In dieser Stadt existieren keine Straßen mehr – umso komplexer wird der Prozess das Navigationsverhalten der Roboter mit den richtigen Parametern einzustellen und zu modellieren und all die verschiedenen Krankentransporter in einer Weise zu synchronisieren, dass keine gegenseitigen Behinderungen auftreten. Die Voraussetzungen und Spezifikation dafür zu erfassen ist Ziel dieser Analyse – die sich mit dem Fahr- und dem Aufladevorgang der Roboter, im speziellen den Fahrvorgang mit Patienten, ihren Manövrierfähigkeiten gegenüber Hindernissen, dem Aufnahmevorgang und die Entsendung zum Standort von Patienten beschäftigt.\\

Im weiteren Verlauf des Projektes könnten weitere Gegebenheiten auftreten, an die sich das System anpassen muss. Alle hier getroffenen Annahmen, Modellierungen und Implementierungen könnten damit im Rahmen der späteren Anwendungen revidiert werden.\\

Als Ausgangspunkt stellt das Unternehmen eine unbegrenzte Anzahl von Krankentransportrobotern zur Verfügung, die mit Aktoren, Sensoren und einem Ortungsgerat ausgestattet sind, und im Wechselspiel mit einem Server agieren.\\

Das Primärziel dieser Analyse ist es dabei einen möglichst reibungslosen Ablauf der Krankentransporte für menschliche Individuen zu ermöglichen. Davon würde die Zielgruppe, spezifisch die Bewohner der Stadt, nachhaltig profitieren, indem Verkehrsunfällen, Staus und die Dauer des Fahrens auf ein Minimum reduziert werden – und somit Verletze stets rechtezeitig zur Behandlung im Krankenhaus eintreffen würden. Die Umsetzung des Systems birgt jedoch zahlreiche Herausforderungen, die im Verlaufe dieser Analyse betrachtet werden– werden sie gelöst, könnte langfristig ein einzelnes Unternehmen die gesamte Krankentransportinfrastruktur einer Großstadt übernehmen. Das System wird von sich aus selbstständig und vollständig autonom agieren. Abgesehen davon Sicherheitsregeln einzuhalten, sollten damit keine Vorkenntnisse für Patienten und ihre Begleiter notwendig sein, um den Krankentransportdienst zu nutzen. Gerade für Patienten sollten die Roboter den Fahrvorgang allerdings mit kleinstmöglichem Aufwand und Belastung gestalten.\\\\

Definition des Problembereichs\\
Der Problembereich erweitert sich für diese Analyse auf sechs Punkte, um eine Gesamtkoordination effizient zu gestalten:\\

\begin{enumerate}
	\item Zuallererst muss das automatische Fahren im zweidimensionalen Koordinatensystem zu einem angegebenen Zielpunkt fehlerfrei möglich sein. Ebenfalls steht eine manuelle Steuerung zur Verfügung, um unerwarteten Ereignissen zu begegnen, die allerdings wiederum sinnvoll in das Gesamtsystem integriert werden muss.
	\item Als zweite Herausforderung gilt es einen möglichst reibungs- und unterbrechungslosen Batteriebetrieb zu garantieren, um zeitlich vorrauschendes Fahren zu ermöglichen. Im Falle eines niedrigen Batteriestands also unverzüglich an eine Ladestation anzudocken und aufzuladen, beziehungsweise längere Fahrten nicht anzutreten, für die der Batteriestand nicht ausreicht.
	\item Im dritten Punkt geht es darum den Schaden von Kollisionen zu minimieren; wie diese erkannt, wie auf sie reagiert wird und diese Kollisionen durch vorrauschendes Umfahren von Objekten verhindert werden können. Dies schließt sowohl bewegliche Objekte als auch unbewegliche Objekte mit ein. Im speziellen muss eine Priorisierung stattfinden, die Roboter müssen unterschiedlich priorisiert werden, damit den Krankentransportern ein reibungsloser Fahrtverlauf garantiert wird.
	\item Dieser Punkt beschäftigt sich mit dem Abholvorgang. Jeder Krankentransport benötigt mindesten ein Vehikel das zur Verfügung steht. Jeder dieser Roboter muss unter bestimmten Kriterien wie Nähe und Batteriestand vom Server ausgewählt werden, und es gilt einzugrenzen, inwieweit diese Kriterien Einfluss auf die Entscheidung des Servers nehmen. Wenn dieser Roboter angekommen ist, müssen wiederum Patient und mögliche Helfende benachrichtigt werden.
	\item Außerdem muss der Fahrvorgang mit einem Patienten unterschiedlich zu dem normalen Fahrvorgang des Roboters eingestellt werden. Und wie verhält sich der Roboter, wenn er das Krankenhaus ansteuert? Welche Daten und Fakten werden ausgetauscht, damit dem Patienten nach der Ankunft die schnellstmögliche Behandlung zukommen kann?
\end{enumerate}

An die Punkte anknüpfend wird mit folgenden Annahmen gearbeitet:\\
\begin{enumerate}
	\item Zur Vereinfachung des Fahrverhalten muss jedes Zielobjekt physisch erreichbar sein. Ziele sind dementsprechend keine Hindernisse, das gleiche gilt für Ladestationen.
	\item Roboter können autark vom Server agieren, damit ihre Ziele bestimmen und eine Ladestation anfahren.
	\item Bei den beweglichen Objekten wird davon ausgegangen, dass es sich im Verkehrsbetrieb ausschließlich um andere Roboter handelt. Für die unbeweglichen Objekte ist es notwendig, dass ihre Position von vornerein bekannt ist und sie unverändert bleibt.
\end{enumerate}
