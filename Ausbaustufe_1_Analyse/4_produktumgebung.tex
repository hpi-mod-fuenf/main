\section{Produktumgebung}
\textcolor{blue}{\textit{In diesem Abschnitt wird der geplante Einsatz des zu entwickelnden Produktes beschrieben. Dies umfasst insbesondere die Systemumgebung in der das Produkt eingesetzt werden soll und die Zuordnung der Produktbestandteile zu dieser sowie die nicht-funktionalen Anforderungen. Je präziser diese Angaben sind, desto besser kann das realistische Verhalten des Produktes in Testumgebungen bestimmt werden.}}

\subsection{Systemumgebung}
\textcolor{blue}{\textit{Hier sollten alle wesentlichen Parameter der Systemumgebung beschrieben werden, soweit diese bereits festgelegt ist.}}

\subsubsection{Hardwareumgebung}
\textcolor{blue}{\textit{Angaben über die existierenden oder zu erwartenden Hardwareumgebungen. Je nach Architektur können hier auch mehrere verschiedene Umgebungen relevant sein.}}

\subsubsection{Softwareumgebung}
\textcolor{blue}{\textit{In diesem Abschnitt werden Angaben zur Softwareumgebung des zu entwickelnden Produktes gemacht. Insbesondere das Betriebssystem und zur VerfÜgung stehende Laufzeitumgebungen /Bibliotheken sind wichtig. Andere Systeme mit denen das zu entwickelnde Produkt kooperieren muss, sollten möglichst genau spezifiziert sein.}}

\subsubsection{Ressourcenübersicht}
\textcolor{blue}{\textit{In diesem Abschnitt sollen die vorhandenen bzw. geplanten Software- und Hardware-Ressourcen dokumentiert werden.}}

\begin{figure}[H]
\centering
\includegraphics[width=0.7\textwidth]{img/Verteilungsdiagramm.png}
\end{figure}