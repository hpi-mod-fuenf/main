\section{Produktumgebung}

  \subsection{Systemumgebung}
  Im nachfolgenden Abschnitt werden die bekannten Komponenten des Systems
  und die dazugehörigen Schnittstellen beschrieben. Grundsätzlich besteht
  das System aus mindestens einem \emph{Robot}, hierfür geeigneten
  \emph{Chargern} und einem zentralen \emph{Server}.

    \subsubsection{Hardwareumgebung}
    \paragraph{Server}\label{server}
    \\

    Es existiert ein zentraler \emph{Server}. Dieser \emph{Server} verfügt
    über ausreichende Ressourcen.

    \subparagraph{IServerCore}\label{iservercore}
    \\

    Der \emph{Server} verfügt über einen zentralen Hauptprozessor, über den
    auf alle weiteren Komponenten zugegriffen werden kann. Dies wird der
    Einfachheit halber angenommen.

    \subparagraph{IServerWlanAdapter}\label{iserverwlanadapter}
    \\

    Beim \emph{IServerWlanAdapter} handelt es sich um eine Komponente, mit der der \emp{Server}
    auf das Funknetzwerk zugreifen kann. Über dieses Funknetzwerk können alle
    \emph{Robots} erreicht werden, sodass Nachrichten an sie abgesetzt werden können
    und Nachrichten, die über das Funknetzwerk für den \emp{Server} übertragen wurden,
    empfangen werden.

    \subparagraph{IHospital}\label{ihospital}
    \\

    Der \emph{Server} verfügt auch über eine \emph{IHospital} Komponente, über die
    der Server mit dem Krankenhaus kommunizieren kann. Dies schließt das Senden und
    und das Empfangen von Nachrichten in beiden Richtungen ein. Es ist nicht spezifiziert,
    wie der genaue Zugriff des Krankenhauses auf diese Komponente realisiert wird.

    \paragraph{Robot}\label{robot}
    \\

    Das Transportvehikel \emph{Robot} erfüllt Grundfunktionalitäten wie das
    Anfahren von Zielen, das Umfahren von Hindernissen und das Laden des
    Akkumulators. Nachfolgend werden die zentralen Hardwarekomponenten des
    \emph{Robots} beschrieben.

    \subparagraph{IRobotCore}\label{irobotcore}
    \\

    Der \emph{Robot} verfügt über einen zentralen Hauptprozessor, über den
    auf alle weiteren Komponenten zugegriffen werden kann. Dies wird der
    Einfachheit halber angenommen.

    \subparagraph{IRobotWlanAdapter}\label{irobotwlanadapter}
    \\

    Mit dem \emph{IRobotWlanAdapter} kann der \emph{Robot} auf das Funknetzwerk
    zugreifen. Er kann über diese Komponente Nachrichten, die beispielsweise vom
    zentralen \emph{Server} gesendet wurden, empfangen und Nachrichten versenden.

    \subparagraph{IRobotEngine}\label{irobotengine}
    \\

    Als Antrieb nutzt das Transportvehikel einen omnidirektionalen Antrieb
    mit 3 Motoren, über welchen es sich vorwärts, nach rechts, nach links
    oder durch Drehen um die eigene Achse bewegen lässt. Der Antrieb
    ermöglicht eine Fahrt in verschiedenen Geschwindigkeiten, wobei eine
    nicht näher spezifizierte Grenze nicht überschritten werden kann.

    \subparagraph{IBattery}\label{ibattery}
    \\

    Jeder \emph{Robot} verfügt über einen Akkumulator, der zur
    Energieversorgung dient. Eine ausreichende Ladung des Akkumulators ist
    deshalb zum Betrieb des \emph{Robots} unbedingt erforderlich. Der
    Akkumulator hat eine maximale Ladekapazität und kann über einen
    \emph{Charger} geladen werden. Die genaue Beschaffenheit des
    Akkumulators ist nicht bekannt.

    \subparagraph{INorthStar}\label{inorthstar}
    \\

    Die Komponente \emph{INorthStar} kann ähnlich einem GPS-Modul die
    Position des \emph{Robots} feststellen. Hierbei ist die Komponente in
    der Lage, die Koordinaten der Position in einem zweidimensionalen
    Koodinatensystem und die aktuelle Ausrichtung des Vehikels zu ermitteln.

    \subparagraph{IRSensorDistance}\label{irsensordistance}
    \\

    Das Transportvehikel verfügt über 9 Infarotdistanzsensoren, die an der
    kreisförmigen Außenwand des Vehikels im Abstand von jeweils 40
    angeordnet sind. Über sie ist die Feststellung der Entfernung des
    Vehikels von allen bewegten und unbewegten Objekten in der Umgebung
    möglich.

    \subparagraph{IBumper}\label{ibumper}
    \\

    Bei einer Kollision des \emph{Robots} mit einem anderen Objekt wird
    dieses Ereignis über die Sensorik einer Kollisionserkennung festgestellt, worauf
    weitere Schritte eingeleitet werden können.

    \paragraph{Charger}\label{charger}
    \\

    Es existieren Ladestationen, über die sich die \emph{IBattery} der
    \emph{Robots} vollständig laden lässt. Eine Ladestation ist dabei genau
    einem \emph{Robot} zugeordnet. Zum Laden muss ein \emph{Robot} seine
    Ladestation anfahren, worauf die Ladung sofort beginnt. Jede Ladestation
    verfügt über eine genaue Position.

    \subsubsection{Softwareumgebung}



\subsubsection{Ressourcenübersicht}
    In Abbildung \ref{fig:4-1-3-verteilungsdiagramm} werden die in 4.1.1 und 4.1.2 beschriebenen
    Komponenten und Schnittstellen im Verteilungsdiagramm dargestellt.

    \begin{figure}[H]
      \centering
      \includegraphics[width=1.5\textwidth, angle=90]{img/1-Analyse-4-Produktumgebung}
      \caption{Verteilungsdiagramm}
      \label{fig:4-1-3-verteilungsdiagramm}
    \end{figure}
