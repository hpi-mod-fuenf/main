\section{Produktumgebung}

  \subsection{Systemumgebung}
  Im nachfolgenden Abschnitt werden die bekannten Komponenten des Systems
  und die dazugehörigen Schnittstellen beschrieben. Grundsätzlich besteht
  das System aus mindestens einem \emph{Robot}, hierfür geeigneten
  \emph{Chargern} und einem zentralen \emph{Server}.

    \subsubsection{Hardwareumgebung}
    \paragraph{Server}\label{server}

    Es existiert ein zentraler \emph{Server}. Dieser \emph{Server} verfügt
    über ausreichende Ressourcen.

    \subparagraph{IServerCore}\label{iservercore}

    Der \emph{Server} verfügt über einen zentralen Hauptprozessor, über den
    auf alle weiteren Komponenten zugegriffen werden kann. Dies wird der
    Einfachheit halber angenommen.

    \subparagraph{IServerWlanAdapter}\label{iserverwlanadapter}

    Beim \emph{IServerWlanAdapter} handelt es sich um eine Komponente, mit der der \emp{Server}
    auf das Funknetzwerk zugreifen kann. Über dieses Funknetzwerk können alle
    \emph{Robots} erreicht werden, sodass Nachrichten an sie abgesetzt werden können
    und Nachrichten, die über das Funknetzwerk für den \emp{Server} übertragen wurden,
    empfangen werden können.

    \subparagraph{IHospital}\label{ihospital}

    Der \emph{Server} verfügt auch über eine \emph{IHospital} Komponente, über die
    der Server mit dem Krankenhaus kommunizieren kann. Dies schließt das Senden und
    und das Empfangen von Nachrichten in beiden Richtungen ein. Es ist nicht spezifiziert,
    wie der genaue Zugriff des Krankenhauses auf diese Komponente realisiert wird.

    \paragraph{Robot}\label{robot}

    Das Transportvehikel \emph{Robot} erfüllt Grundfunktionalitäten wie das
    Anfahren von Zielen, das Umfahren von Hindernissen und das Laden des
    Akkumulators. Nachfolgend werden die zentralen Hardwarekomponenten des
    \emph{Robots} beschrieben.

    \subparagraph{IRobotCore}\label{irobotcore}

    Der \emph{Robot} verfügt über einen zentralen Hauptprozessor, über den
    auf alle weiteren Komponenten zugegriffen werden kann. Dies wird der
    Einfachheit halber angenommen.

    \subparagraph{IRobotWlanAdapter}\label{irobotwlanadapter}

    Mit dem \emph{IRobotWlanAdapter} kann der \emph{Robot} auf das Funknetzwerk
    zugreifen. Er kann über diese Komponente Nachrichten, die beispielsweise vom
    zentralen \emph{Server} gesendet wurden, empfangen und Nachrichten versenden.

    \subparagraph{IRobotEngine}\label{irobotengine}

    Als Antrieb nutzt das Transportvehikel einen omnidirektionalen Antrieb
    mit 3 Motoren, über welchen es sich vorwärts, nach rechts, nach links
    oder durch Drehen um die eigene Achse bewegen lässt. Der Antrieb
    ermöglicht eine Fahrt in verschiedenen Geschwindigkeiten.

    \subparagraph{IBattery}\label{ibattery}

    Jeder \emph{Robot} verfügt über einen Akkumulator, der zur
    Energieversorgung dient. Eine ausreichende Ladung des Akkumulators ist
    deshalb zum Betrieb des \emph{Robots} unbedingt erforderlich. Der
    Akkumulator hat eine maximale Ladekapazität und kann über einen
    \emph{Charger} geladen werden. Die genaue Beschaffenheit des
    Akkumulators ist nicht bekannt.

    \subparagraph{INorthStar}\label{inorthstar}

    Die Komponente \emph{INorthStar} kann ähnlich einem GPS-Modul die
    Position des \emph{Robots} feststellen. Hierbei ist die Komponente in
    der Lage, die Koordinaten der Position in einem zweidimensionalen
    Koodinatensystem und die aktuelle Ausrichtung des Vehikels zu ermitteln.

    \subparagraph{IRSensorDistance}\label{irsensordistance}

    Das Transportvehikel verfügt über 9 Infarotdistanzsensoren, die an der
    kreisförmigen Außenwand des Vehikels im Abstand von jeweils 40
    angeordnet sind. Über sie ist die Feststellung der Entfernung des
    Vehikels von allen bewegten und unbewegten Objekten in der Umgebung
    möglich.

    \subparagraph{IBumper}\label{ibumper}

    Bei einer Kollision des \emph{Robots} mit einem anderen Objekt wird
    dieses Ereignis über die Sensorik einer Kollisionserkennung festgestellt, worauf
    weitere Schritte eingeleitet werden können.

    \paragraph{Charger}\label{charger}

    Es existieren Ladestationen, über die sich die \emph{IBattery} der
    \emph{Robots} vollständig laden lässt. Eine Ladestation ist dabei genau
    einem \emph{Robot} zugeordnet. Zum Laden muss ein \emph{Robot} seine
    Ladestation anfahren, worauf die Ladung sofort beginnt. Jede Ladestation
    verfügt über eine genaue Position.

    \subsubsection{Softwareumgebung}

    \paragraph{Server}\label{server}
    		Auf dem \emph{Server} läuft, da nicht anders angegeben, ein Standard-Betriebssystem. Darauf läuft eine Laufzeitumgebung, die die benötigten Methoden zur Kommunikation mit den \emph{Robots} bereitstellt. Die wichtigste (und einzige bisher spezifizierte) Methode, ist die CallRobot() Methode um einen \emph{Robot} anzurufen und einen Datenaustausch herzustellen.
    	\subparagraph{IServerWlanAdapter}\label{iserverwlanadapter}
    		Der \emph{IServerWlanAdapter} ist die Komponente des \emph{Servers}, mit der er die Verbindung zum \emph{Robot} herstellt. Hierüber werden Nachrichten ausgetauscht. Dementsprechend stehen Methoden zum Senden einer Message, registrieren eines IMessageHandlers, auslesen von NetworkIDs usw. bereit.
    	\subparagraph{IHospital}\label{ihospital}
    		Bei \emph{IHospital} handelt es sich um die \emph{Server} Komponente, die die Kommunikation mit dem \emph{Hospital} gewährleistet. Dabei werden zum einen Methoden bereitgestellt, die das \emph{Hospital} über ein Interface aufrufen kann, um Aufträge zu verteilen und dem \emph{Hospital} Informationen über den aktuellen Stand des Auftrags mitzuteilen (getPatientAt(), patientOnBoard() sowie patientArrived()), und zum anderen kann die Komponente über ein Interface Methoden des \emph{Hospitals} aufrufen, um diesem Informationen zu übermitteln (informRobotArrivedAtPatient() und informRobotArrivedAtHospital()).
    \paragraph{Robot}\label{robot}
    		Nachfolgend werden die zentralen Softwareschnittstellen des \emph{Robots} beschrieben.
    	\subpragraph{IRobotCore}\label{irobotcore}
    		Auf der zentralen Recheneinheit des \emph{Robots}, dem \emph{IRobotCore}, läuft eine JavaRuntimeEnvironment, in der sich die gesamte Steuerung und die Verwendung der Komponenten und Schnittstellen abspielt. Die einzelnen Komponenten mit ihren Methoden werden im folgenden näher erläutert.
    	\subparagraph{IRSensorDistance}\label{irsensordistance}
    		Die \emph{IRSensorDistance} Komponente stellt drei Methoden bereit, mit denen die Distanz und der Winkel des entdeckten Objekts und der an der Entdeckung beteiligte Infrarotsensor erfasst und in Arrays gespeichert werden.
    	\subparagraph{IDistanceSensor}\label{idistancesensor}
    		Die besagten Arrays kann der \emph{IDistanceSensor} auslesen. Er hat dazu ebenfalls drei Methoden. getIRDistances() gibt ein Array mit allen erfassten Objekten zurück, getIRDistancesInRange() gibt ebenfalls ein Array zurück, das allerdings nur die Objekte in einem gewissen Abstand enthält und getNearestIRDistances() gibt nur das nächste Objekt zurück.
    	\subparagraph{INorthStar}\label{inorthstar}
    		Die Komponente \emph{INorthStar} ist für die Positionierung zuständig und greift dabei auf ein Device zur Standortbestimmung zurück. Sie hat zwei Methoden. Eine liest die aktuelle Position aus, die andere die aktuelle Ausrichtung. Die Position besteht dabei aus zwei float-Werten, einer x- und einer y-Koordinate.
    	\subparagraph{IRobotWlanAdapter}\label{irobotwlanadapter}
    		Bei \emph{IRobotWlanAdapter} handelt es sich um die Komponente, mit der auf Seite des \emph{Robots} die Kommunikation zwischen \emph{Robot} und \emph{Server} ermöglicht wird. Entsprechend gibt es hier die gleichen Methoden zur Kommunikation, die auch die \emph{IServerWlanAdapter} Komponente bereitstellt.
    	\subparagraph{IBumperHandler}\label{ibumperhandler}
    		\emph{IBumperHandler} ist die Komponente zum Umgang mit Zusammenstößen. Die Kollisionserkennung IBumper registriert einen Aufprall und die \emph{IBumperHandler} Komponente stellt zwei Methoden zum Umgang mit dem Aufprall bereit.
    	\subparagraph{IDrive}\label{idrive}
    		Die Bewegungssteuerung des \emph{Robots} heißt \emph{IDrive}. Sie stellt vier Methoden bereit. Die Methoden driveToPosition() und driveToPositionCautiously() erwarten beide eine Position, eine Geschwindigkeit und einen ArrivalHandler, der eine Methode zur Meldung aufruft, wenn der \emph{Robot} am Ziel angekommen ist. Die Methoden sind beide dafür da, ein gegebenes Ziel anzufahren, wobei bei der Zweiten die Höchstgeschwindigkeit geringer ist. Die höchstgeschwindigkeit für Cautiously, Regular und Fast Methoden stellt die \emph{IDrive} Komponente bereit, wobei Cautiously ≤ Regular ≤ Fast gilt. Die anderen beiden Methoden, drive() und driveCautiously() unterscheiden sich ebenfalls nur in der Höchstgeschwindigkeit und ermöglichen ein manuelles Fahren. Dabei erwarten sie die Drehung des \emph{Robots} als float-Wert sowie die Vorwärts- und die Seitwärtsgeschwindigkeit, ebenfalls als float-Wert.
    	\subparagraph{IBattery}\label{ibattery}
    		Bei \emph{IBattery} handelt es sich um die Akkusteuerung des \emph{Robots}. Sie stellt zwei Methoden bereit. getBatteryLevel() gibt den aktuellen Akkuladestand als float-Wert zurück, getChargingPosition() gibt die Position des zugeordneten Chargers zurück.


\pagebreak
\subsubsection{Ressourcenübersicht}
    In Abbildung \ref{fig:4-1-3-verteilungsdiagramm} werden die in 4.1.1 und 4.1.2 beschriebenen
    Komponenten und Schnittstellen im Verteilungsdiagramm dargestellt.

    \begin{figure}[H]
      \centering
      \includegraphics[width=1.2\textwidth, angle=90]{img/1-Analyse-4-Produktumgebung}
      \caption{Verteilungsdiagramm}
      \label{fig:4-1-3-verteilungsdiagramm}
    \end{figure}
